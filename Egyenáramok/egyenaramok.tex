\documentclass [12pt,a4paper] {article}
\usepackage [utf8] {inputenc}
\usepackage [magyar] {babel}
\usepackage [T1] {fontenc}
\usepackage{amsmath}
\usepackage{amsfonts}
\usepackage{amssymb}
\usepackage[left=2cm,right=2cm,top=2cm,bottom=2cm] {geometry}
\author{Gáspár Bernát}
\title{Egyenáramok}

\begin{document}
\maketitle
\section{Áramerősség}
\underline{áram:} töltéshordozok egyirányú rendezett áramlása \newline
\underline{feltétele:} a vezető két vége között potenciál különbség legyen (feszültség) \newline
\underline{iránya:} pozitív töltések haladási iránya \newline\newline

Áramerősség: $I = \dfrac{Q}{t} \hspace{10mm} [I] = \dfrac{C}{s} = a$ \newline\newline
Az áramerősséget (I) jellemezhetjük a vezető teljes keresztmetszetén egységnyi idő alatt áthaladó összes töltés mennyiségével (Q) és az idő (t) hányadosaként. \newline
Ha az áram iránya és erőssége állandó az időben, akkor I = áll. azaz egyenáram

\section{Ohm-törvény}
az áramerősség a vezeték két rögzített pontja között mérhető feszültséggel egyenesen arányos\newline\newline
ezek alapján a törvény: $R = \dfrac{U}{I} \hspace{10mm} [R] = \dfrac{V}{A} = \Omega$ ahol R az adott vezetékszakaszra jellemző ellenállás \newline \newline
ezen kívül számolhatunk ellenállást a keresztmetszet nagysága és egy adott vezetékhossz szakaszának ismeretében \newline\newline
$R = \rho\dfrac{l}{A} {\hspace{10mm} \rho = \dfrac{RA}{l}}$ itt a $\rho$ egy arányossági tényező, \underline{fajlagos ellenállás}\newline
az SI egysége: $\dfrac{\Omega mm^2}{m}$ azért nem $m^2$, mert az gyakorlatban túl nagy érték (vezeték keresztmetszetről van szó)\newline\newline
Megjegyzés: a vezeték általában henger alakú szóval kör a keresztmetszete (A) és a kör területe pedig $r^2\pi$, ahol az r a sugár, aminek a kétszerese az átmérő, ez általában d

\section{Soros kapcsolás}
Az ellenállásokat sorban, elágazás nélkül kötjük be \newline
$I = I_1 = I_2 = I_n$ töltésmegmaradás \newline
$U_k = U_1 + U_2 + U_n$ \newline
Eredő ellenállás: $R_e = \dfrac{U_k}{I} = \dfrac{U_1 + U_2 + U_n}{I} = R_1 + R_2 + R_n$

\section{Párhuzamos kapcsolás}
Az ellenállásokat párhuzamosan kapcsoljuk be egymáshoz képest \newline
$I = I_1 + I_2 + I_n$ \newline
$U_k = U_1 = U_2 = U_n$ \newline
Eredő ellenállás: $R_e = \dfrac{U_k}{I} = \dfrac{U}{I_1 + I_2 + I_n} \rightarrow \dfrac{I_1 + I_2 + I_n}{U} = \dfrac{I_1}{U} + \dfrac{I_2}{U} + \dfrac{I_n}{U} = \dfrac{1}{R_1} + \dfrac{1}{R_2} + \dfrac{1}{R_n} = \dfrac{1}{R_e}$

\section{Elektromos munka és teljesítmény}
Egymásból jól kifejezhetőek \newline
$W = U\cdot I\cdot t \hspace{10mm} [W] = V\cdot A\cdot s = J$ \newline
$P = U\cdot I \hspace{10mm} [P] = V\cdot A = W$(Watt, nem munka, de sajnos az is W, ugye mint az előző sorban) \newline
Ezekből jól észrevehtő, hogy $W = P\cdot t$ \newline
Esetleg felírhatjuk, hogy $P = \dfrac{W}{t} = \dfrac{U\cdot I\cdot t}{t} = U\cdot I$ szóval valóban visszakapjuk \newline
A munka (W) lényegében megegyezik az energiával, legalábbis a mértékegysége, így a hőmérsékletes feladatoknál is használhatjuk a munkát

\end{document}