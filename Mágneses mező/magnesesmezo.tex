\documentclass [12pt,a4paper] {article}
\usepackage [utf8] {inputenc}
\usepackage [magyar] {babel}
\usepackage [T1] {fontenc}
\usepackage{amsmath}
\usepackage{amsfonts}
\usepackage{amssymb}
\usepackage[left=2cm,right=2cm,top=2cm,bottom=2cm] {geometry}
\author{Gáspár Bernát}
\title{Mágneses mező}

\begin{document}
\maketitle
\section{Mágneses indukció}
A mágneses indukció a mágneses tér erősségére jellemző vektormennyiség (mágneses indukció vektor)\newline
Jele: \underline{B} \hspace{5mm} $[\underline{B}] = T$ (Tesla) \newline
Nagysága függ: a vonalak sűrűségétől\newline
Az iránya: a húzott érintő iránya\newline

\section{Mágneses fluxus}
A fluxus egy erőtérnek a felületén való áthatolását jellemzi \newline
$\phi = \underline{B} \cdot A \rightarrow [\phi] = T \cdot m^2 = Wb$ (Weber) \newline
valójában ez a képlet hiányos, mivel befolyásolja a fluxust a felület vonalakkal bezárt szöge \newline
$\phi = \underline{B} \cdot A \cdot sin\alpha$ \newline
Ha a felület derékszöget zár be a vonalakkal, akkor $\phi \rightarrow \underline{B} \angle A \rightarrow \phi = max$ \newline
Ha a felület párhuzamos a vonalakkal, akkor $\underline{B} || A \rightarrow \phi = 0$

\section{Forgatónyomaték}
$\underline{M} = N \cdot A \cdot I \cdot \underline{B}$ (N $\rightarrow$ menetszám) \newline
$\underline{M} = 0$, ha $\underline{B} \angle A \rightarrow (\phi max)$ \newline
$\underline{M} = max$, ha $\underline{B} || A \rightarrow (\phi 0)$ \newline
Így láthatjuk jól, hogy a fluxus pont ellentétes a menetszámmal

\section{Egyenes vezető mágneses tere}
$\underline{B} = \mu_0 \mu_r \dfrac{NI}{l}$ tekercs tere \newline
A végtelen hosszú, egyenes vezető körül a mágneses indukció nagysága egyenesen arányos a vezetőben folyó áram erősségével, és fordítottan arányos a vezetőtől mért távolsággal: \newline
$\underline{B} = \dfrac{I}{2r\pi} \mu_0$ hosszú egyenes vezető\newline
$\mu_0 = 4\pi \cdot 10^-7 \dfrac{V_0}{A m}$ \newline
Itt zárt koncentrikus körök vannak (azt jelenti, hogy azonos a középpontjuk, de más a sugaruk)

\section{Mágneses megosztás}
A ferromágneses anyagok vonzó képessége megoszlik a térben lévő ferromágneses anyagokkal. Például: több szög vonzza egymást, ha 1 szög mágneshez van érintve

\section{Elektromágnes}
Elektromágnes: az egyenáram hatása által vezetőben létrehozott mágneses mező. Az erőssége az áramerősséggel (A), a ferromágneses vasmag tulajdonságával (mű) illetve a menetszámmal szabályozható 

\end{document}

